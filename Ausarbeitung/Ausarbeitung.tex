% Diese Zeile bitte -nicht- aendern.
\documentclass[course=erap]{aspdoc}

%%%%%%%%%%%%%%%%%%%%%%%%%%%%%%%%%
%% TODO: Ersetzen Sie in den folgenden Zeilen die entsprechenden -Texte-
%% mit den richtigen Werten.
\newcommand{\theGroup}{134} % Beispiel: 42
\newcommand{\theNumber}{A404} % Beispiel: A123
\author{Arman Habibi \and Larissa Manalil \and Andrei Stoica}
\date{Wintersemester 2022/23} % Beispiel: Wintersemester 2019/20
%%%%%%%%%%%%%%%%%%%%%%%%%%%%%%%%%

% Diese Zeile bitte -nicht- aendern.
\title{Gruppe \theGroup{} -- Abgabe zu Aufgabe \theNumber}

\begin{document}
\maketitle

\section{Einleitung}

Die von David A. Huffman, im jahr 1952, entwickelte Huffmankodierung dient zur verlustfreien Datenkompression.
Um Redundanz zu minimieren, werden, nach einer Häufigkeitsanalyse der Symbole, diese in einem Binärbaum angeordnet, sodass der entstehende Binärkode, mit welchem das Symbol im Baum zu erreichen ist, präfixfrei ist.
Dies bedeutet, dass kein Binärkode Teil eines anderen sein darf und ermöglicht es die Daten ohne Trennzeichen zu speichern. Die Symbole, die häufiger in der Quelldatei auftauchen, befinden sich weiter oben im Baum und brauchen daher auch weniger Bits zum abspeichern. 
Das Verfahren ist soweit Optimal, da es keinen anderen symbolbasierten Weg gibt, der einen kürzeren Code erzeugen könnte, insofern die Häufigkeiten der Symbole bekannt sind. \cite{4051119}
Im folgenden gilt es diesen Algorithmus in der Programmiersprache C zu implementieren und ein Rahmenprogramm zu erstellen, welches es ermöglicht Dateien zu komprimieren und zu dekodieren.

\section{Lösungsansatz}


% TODO: Je nach Aufgabenstellung einen der Begriffe wählen
\section{Korrektheit/Genauigkeit}


\section{Performanzanalyse}


\section{Zusammenfassung und Ausblick}

% TODO: Fuegen Sie Ihre Quellen der Datei Ausarbeitung.bib hinzu
% Referenzieren Sie diese dann mit \cite{}.
% Beispiel: CR2 ist ein Register der x86-Architektur~\cite{intel2017man}.
\bibliographystyle{plain}
\bibliography{Ausarbeitung}{}

\end{document}
