% Diese Zeile bitte -nicht- aendern.
\documentclass[course=erap]{aspdoc}

%%%%%%%%%%%%%%%%%%%%%%%%%%%%%%%%%
%% TODO: Ersetzen Sie in den folgenden Zeilen die entsprechenden -Texte-
%% mit den richtigen Werten.
\newcommand{\theGroup}{134} % Beispiel: 42
\newcommand{\theNumber}{A404} % Beispiel: A123
\author{Arman Habibi \and Larissa Manalil \and Andrei Stoica}
\date{Wintersemester 2022/23} % Beispiel: Wintersemester 2019/20
%%%%%%%%%%%%%%%%%%%%%%%%%%%%%%%%%

% Diese Zeile bitte -nicht- aendern.
\title{Gruppe \theGroup{} -- Abgabe zu Aufgabe \theNumber}

\begin{document}
\maketitle

\section{Einleitung}

Die von David A. Huffman, im jahr 1952, entwickelte Huffmankodierung dient zur verlustfreien Datenkompression.
Um Redundanz zu minimieren, werden, nach einer Häufigkeitsanalyse der Symbole, diese in einem Binärbaum angeordnet, sodass der entstehende Binärkode, mit welchem das Symbol im Baum zu erreichen ist, präfixfrei ist.
Dies bedeutet, dass kein Binärkode Teil eines anderen sein darf und ermöglicht es die Daten ohne Trennzeichen zu speichern. Die Symbole, die häufiger in der Quelldatei auftauchen, befinden sich weiter oben im Baum und brauchen daher auch weniger Bits zum abspeichern. 
Das Verfahren ist soweit Optimal, da es keinen anderen symbolbasierten Weg gibt, der einen kürzeren Code erzeugen könnte, insofern die Häufigkeiten der Symbole bekannt sind. \cite{4051119}
Im folgenden gilt es diesen Algorithmus in der Programmiersprache C zu implementieren und ein Rahmenprogramm zu erstellen, welches es ermöglicht Dateien zu komprimieren und zu dekodieren.

\section{Lösungsansatz}

\subsection{Encode}

Zuerst wird die Häufigkeitsanalyse der Symbole durchgeführt. Dafür wird ein Array mit Pointern auf Nodes für jedes ASCII Zeichen angelegt, welche die Frequenz des Zeichens speichern, sowie das Zeichen für die spätere Verwendung. Das Zählen erfolgt hierbei mit einer Effizienz von $\mathcal{O}(1)$, da der Array-Index dem ASCII-Code entspricht.
Zusätzlich wird Speicher gespart, indem die Nodes erst erstellt werden sobald das zugehörige ASCII Zeichen das erste mal gelesen wird.

Als Beispiel wird der String $ABRACADABRAB$ komprimiert, welcher die Folgenden Häufigkeiten enthält. Nicht vorhandene Zeichen wurden vernachlässigt.

\begin{center}
    \begin{tabular}{ |c|c|c|c|c| } 
     \hline
     A & B & C & D & R \\ 
     \hline
     5 & 3 & 1 & 1 & 2 \\ 
     \hline
    \end{tabular}
\end{center}

Anschließend wird ein Min-Heap erstellt und die erstellten Nodes darin eingefügt. Hier fängt der von Huffman entwickelte Algorithmus an. \cite{10.1145/3342555}
Solange mehr als ein Node im Min-Heap ist, werden die beiden Nodes mit der geringsten Frequenz aus dem Min-Heap entfernt und an einem neuen Node angehängt (mit einem Null Zeichen), welcher die Summe der beiden Frequenzen als Wert erhält. Dieser wird wieder in den Min-Heap eingefügt.
Am Ende bleibt nur noch ein Node im Min-Heap übrig, welcher den fertigen Huffman-Baum enthält.

\subsection{Decode}

% TODO: Je nach Aufgabenstellung einen der Begriffe wählen
\section{Korrektheit}

Es wird auf die Korrektheit des Huffman Algorithmus eingegangen, da er verlustfreie Komprimierung ermöglicht und somit Genauigkeit als Thema unangemessen wäre.
Um das Problem des optimalen und präfixfreien Codes zu lösen sind zunächst einige Definitionen notwendig.

\subsection{Kodierungstheorie}

Das \textit{Quellalphabet} der Länge $n$ entspricht der Anzahl an verschiedenen Zeichen die in der Quelldatei vorkommen. In diesem Fall entspricht das maximale Quellalphabet der Menge an ASCII Zeichen.
Jedem Zeichen wird ein \textit{Gewicht} $w$ zugeordnet, welches hier der absoluten Häufigkeit des Zeichens in der Quelldatei entspricht. Die Menge der Gewichte ist somit definiert als $W = \langle w_i > 0 \mid 0 \le i < n \rangle $.
Des Weiteren wird auch ein \textit{Codealphabet} definiert, welches die Binäre Menge $\lbrace0, 1\rbrace$ ist, mit dem die Quelldatei komprimiert wird.
Der \textit{Code} ordnet jedem Zeichen $i$ des Quelllphabets ein Codewort aus dem Codealphabet der Länge $l_i$ zu und ist definiert als die Menge $T = \langle l_i > 0 \mid 0 \le i < n \rangle$.

Die Kraft-McMillan-Ungleichung gilt als notwendige Bedingung für präfixfreie Codes.
$$\mathcal{K}(T) = \sum_{i=0}^{n-1} 2^{-l_i} \le 1$$
Codes die diese Bedingung erfüllen werden \textit{zulässig} genannt, dies bedeutet dass eine Belegung an Codewörten existiert, die präfixfrei sind.
Eine Menge an Codewörten ist für einen gegebenen Code \textit{gültig}, insofern die Längen $\langle l_i \rangle$ übereinstimmen und kein Codewort präfix eines Anderen ist.
Es ist somit möglich zu sagen, dass eine präfixfreier Code determiniert wurde, sobald ein zulässiger Code gefunden wurde, ohne die spezifischen Codewörter zu betrachten, da man weiß, dass eine gültige Belegung existiert.

Zuletzt sind noch die \textit{Kosten} $C$ eines Codes zu definieren, welches der Länge der kodierten Nachricht entspricht.
$$C(W,T) = C(\langle w_i \rangle,\langle l_i \rangle) = \sum_{i=0}^{n-1} w_i \cdot l_i $$
Ein Code $T = \langle l_i > 0 \mid 0 \le i < n \rangle$ gilt als optimal, falls für jeden anderen zulässigen Code T' der Länge $n$ gilt: $C(W,T) \le C(W,T^{\prime})$. Dies bedeutet auch, dass mehrere optimale Codes für eine gegebene Quelldatei existieren können.
Für einen optimalen Code muss die Kraft-McMillan-Summe 1 ergeben, da bei kleineren Werten immer ein Codewort gekürzt werden kann, während größere Werte von Anfang an nicht zulässig sind.

\subsection{Optimalität}

Um zu zeigen, dass der Huffman Algorithmus tatsächlich optimal ist, muss zunächst die \textit{geschwister Eigenschaft} betrachtet werden.
Diese besagt, dass ein optimaler Code existiert, bei dem die zwei Zeichen mit dem geringsten Gewicht geschwister sind, also einen Elternknoten im zugehörigen Binärbaum teilen.

Angenommen es gibt einen optimalen Code $T = \langle l_i \rangle $ mit einer Häufigkeitsverteilung $W = \langle w_i \rangle $ auf $n$ Zeichen.
Da jeder nicht-Blatt Knoten zwei Kinder haben muss, existieren zwei Blatt Knoten mit der höchsten Tiefe $ L = max_i l_i$. Ohne Beschränkung der Allgemeinheit sind diese Knoten $w_{n-1}$ und $w_{n-2}$ diejenigen mit den geringsten Gewicht. Falls sie die selbe Tiefe haben, kann der Baum nun umstrukturiert werden, um die beiden Knoten zu Geschwistern zu machen, ohne die Kosten des Codes zu verändern.

Der Beweis der Optimalität erfolgt durch Induktion über der Anzahl an Zeichen $n$. \cite{HufProof}
Als Induktionsbasis gilt n = 2, da es hier nur den optimalen Code $ \langle l_i \rangle = \langle 1, 1 \rangle $ geben kann, nähmlich die beiden Codewörter 0 und 1.
Es wird angenommen, dass die Aussage für $(n-1)$ Zeichen valide ist um sie für $n$ Zeichen zu beweisen.
Man nimmt an es gibt die Gewichte $W = \langle w_0,..., w_{n-2}, w_{n-1} \rangle $, sodass für die Gewichte ohne Beschränkung der Allgemeinheit gilt $w_o \ge w_1 \ge ... \ge w_{n-1} > 0$.

Ausgehend von der geschwister Eigenschaft kann man nun sagen, dass es einen Optimalen Code $T = \langle l_0,..., l_{n-2}, l_{n-1} \rangle$ gibt, wo die Gewichte $w_{n-1}$ und $w_{n-2}$ geschwister sind. Es gilt nun zu Beweisen, dass der durch Huffman hergeleitete Code $H$ dieselben Kosten hat.

Wenn man nun die beiden Knoten $w_{n-2}$ und $w_{n-1}$ entfernt und dem Elternknoten die Summe der Gewichte zuordnet, erhält man den Code $T^{\prime}$ mit $(n-1)$ Zeichen und den Gewichten $W^{\prime} = \langle w_0,..., w_{n-3}, w_{n-2} + w_{n-1} \rangle$.
Das Selbe geschieht nun für den Huffman Code $H^{\prime}$ mit den gleichen Gewichten $W_H^{\prime} = \langle w_0,..., w_{n-3}, w_{n-2} + w_{n-1} \rangle $ per Definition vom Huffman Algorithmus.
Es gilt per Induktionsannahme: 
$$C(W_H^{\prime}, H^{\prime}) = C(W^\prime, T^{\prime})$$
Ebenfalls gilt,
$$ C(W^\prime, T^{\prime}) = C(W, T) - (w_{n-2} + w_{n-1})$$
da die Summe der Knoten $w_{n-2}$ und $w_{n-1}$ nun auf der Tiefe $L-1$ liegt.
Das selbe gilt wiederum für den Huffman Code:
$$ C(W_H^\prime, H^{\prime}) = C(W_H, H) - (w_{n-2} + w_{n-1})$$
Diese drei Gleichungen resultieren in folgendem Beweis.
$$ C(W_H, H) = C(W_H^\prime, H^{\prime}) + w_{n-2} + w_{n-1} = C(W^\prime, T^{\prime}) + w_{n-2} + w_{n-1} = C(W, T) $$

\subsection{Komprimierungseffektivität}

Claude Shannon definierte, 1948, die \textit{Entropie} einer Menge an Gewichten folgendermaßen. \cite{6773024}
$$ \mathcal{H}(W) = - \sum_{i=0}^{n-1} w_i \cdot log_2 \frac{w_i}{m} $$
Sie lässt sich aufteilen auf die Summe der Gewichte $m = \sum_{i=0}^{n-1} w_i$ und den entropischen Kosten (in Bits) $-log_2 (w_i/m) = log_2 (m/w_i) $ einer Instanz eines Zeichens, welches mit einer Wahrscheinlichkeit von $(w_i/m)$ auftritt.
Die Entropie $\mathcal{H}(W)$ gibt die geringste mögliche Länge in Bits an, die der komprimierte Code haben kann, wenn man die Häufigkeitsverteilung mit einbezieht.
Der \textit{relative Effektivitätsverlust} $\mathcal{E}$ eines zulässigen Codes $T$ und der zugehörigen Gewichte $W$ ist durch folgende Gleichung zu berechnen.
$$ \mathcal{E}(W, T) = - \frac{C(W,T) - \mathcal{H}(W)}{\mathcal{H}(W)} $$
Ein relativer Effektivitätsverlust von 0 bedeutet, dass die Zeichen mit ihren entropischen Kosten komprimiert werden, wohingegend bei größeren Werten, diese nicht erreicht werden.

Die Huffmankodierung kommt, in den meisten Situationen, den entropischen Kosten sehr nahe und resultiert daher in Effektivitätsverlusten, die wenige Prozente über dem Optimalfall liegen.
Jedoch gibt es auch einige Fälle, wo der Huffman Code einen großen relativen Effektivitätsverlust hat, insbesonderen, wenn $w_0$ im Vergleich zur Summe der anderen Zeichen sehr groß wird und $w_0 / m$ nahe an die 1 kommt.
Zum Beispiel hat die Gewichtmenge $W = \langle 96, 1, 1, 1, 1 \rangle$ eine Entropie von $\mathcal{H}(W) =
32.2$ Bits , ein zugehöriger optimaler Code $T = \langle 1, 3, 3, 3, 3 \rangle$ mit Kosten $C(W ,T) = 108 $ Bits, und somit einen
relativen Effektivitätsverlust von $\mathcal{E}(W ,T ) = 235\%$. Ein optimaler Code bedeutet somit nicht unbedingt optimale Komprimierung.

Robert Gallager \cite{1055959} zeigte folgende Beziehung, zwischen einem optimalen Code $T = \langle l_i \rangle $ und der Gewichtsmenge $W = \langle w_i \rangle$ mit der Summe $\sum_{i=0}^{n-1} w_i = m$, mit der sich eine obere Schranke für den relativen Effektivitätsverlust berechnen lässt.

$$ C(W, T) - \mathcal{H}(W) \le \Bigl\lbrace
    \begin{array}{l}
          w_0 + 0.086 \cdot m \textrm{, wenn $w_0$ < m/2}\\
          w_0 \textrm{, wenn $w_0 \ge$ m/2}\\
    \end{array} $$

\section{Performanzanalyse}


\section{Zusammenfassung und Ausblick}

% TODO: Fuegen Sie Ihre Quellen der Datei Ausarbeitung.bib hinzu
% Referenzieren Sie diese dann mit \cite{}.
% Beispiel: CR2 ist ein Register der x86-Architektur~\cite{intel2017man}.
\bibliographystyle{plain}
\bibliography{Ausarbeitung}{}

\end{document}
